\documentclass{article}

\usepackage[T2A]{fontenc}
\usepackage[utf8]{inputenc} 
\usepackage[english,russian]{babel} 
\usepackage{graphicx} 
\usepackage{amsmath}
\usepackage{amssymb}
\usepackage{cancel}
\usepackage{amsfonts}
\usepackage{titlesec}
\usepackage{titling} 
\usepackage{geometry}
\usepackage{pgfplots}
\usepackage{esint}
\pgfplotsset{compat=1.9}


\titleformat{\section}
{\normalfont\Large\bfseries}{\arabic{section}}{1em}{}
\titleformat{\subsection}
{\normalfont\large\bfseries}{}{1em}{}


\setlength{\droptitle}{-3em} 

\title{\vspace{-1cm}Дифференциальная геометрия и топология \\ Домашняя работа №1}
\author{Чилеше Абрахам}
\date{Группа: Б9122-02.03.01сцт}

\geometry{a4paper, margin=2cm}

\begin{document}
	
	\maketitle

	\section*{Доказать являются ли вектора базисами}

        \[
        a_1 \begin{pmatrix} 1 \\ 1 \\ 0 \end{pmatrix}, \quad a_2 \begin{pmatrix} 1 \\ 0 \\ 1 \end{pmatrix}, \quad a_3 \begin{pmatrix} 0 \\ 1 \\ 1 \end{pmatrix}
        \]
        
        \[
        b_1 \begin{pmatrix} 1 \\ 2 \\ 2 \end{pmatrix}, \quad b_2 \begin{pmatrix} 2 \\ 1 \\ -2 \end{pmatrix}, \quad b_3 \begin{pmatrix} 2 \\ 2 \\ 1 \end{pmatrix}
        \]
        \vspace{0.3cm}
            
        \[
        \mathsf{A:} \quad \begin{vmatrix} 
        1 & 1 & 0 \\
        1 & 0 & 1 \\
        0 & 1 & 1 
        \end{vmatrix} = \mathsf{1(0-1) - 1(1-0) + 0 = -1 - 1 = -2 \neq 0  \text{ Линейно независим}}
        \]
        
        \vspace{0.2cm}
        \[
        \mathsf{B:} \quad \begin{vmatrix}
        1 & 2 & 2 \\
        2 & 1 & -2 \\
        2 & 2 & 1
        \end{vmatrix} = \mathsf{1(1+4)-2(2+4)+2(4-2) = 5-12+4 = -3 \neq 0  \text{ Линейно независим}}
        \]

         \subsection{Ответ: $\textbf{Являются базисами}$}
        
        \vspace{0.8cm}
       
         \section{Найти матрицу перехода от базиса $\{a\}$ к базису $\{b\}$} 
        
        Найти матрицы перехода от \( a \) к \( b \) \\ \\
        \(   Formula: A^{-1} = \frac{1}{|A|} \cdot A^T \)
    
        \[
        C = \left( A^{-1} B \right)^T = 
        \left( \begin{pmatrix} 
        1 & 1 & 0 \\
        1 & 0 & 1 \\
        0 & 1 & 1
        \end{pmatrix}^{-1} 
        \cdot
        \begin{pmatrix} 
        1 & 2 & 2 \\
        2 & 1 & -2 \\
        2 & 2 & 1
        \end{pmatrix} \right)^T
        \]
        
        \[
       C = \frac{1}{2} 
        \begin{pmatrix} 
        1 & 1 & 3 \\
        1 & 3 & 1 \\
        -1 & 5 & -3
        \end{pmatrix}
        \]
        
        \vspace{0.6cm}
        \section{Найти координаты вектора $X = \begin{pmatrix} -1 \\ 2 \\ 7 \end{pmatrix}$ в базисах $\{a\}$ и $\{b\}$} 
        
        Для \( A \):
        
        \[
        A\begin{cases} 
        3 = d_{1} + d_{2} \\
        -1 = d_{1} + d_{3} \\
        7 = d_{2} + d_{3}
        \end{cases}
        \]
        
        \[
        D = \frac{1}{2} \begin{pmatrix} -5 \\ 1 \\ 3 \end{pmatrix}
        \]
        
        Для \( B \):
        
        \[
        \begin{cases}
        3 = w_1 + 2w_2 + 2w_3 \\
        -1 = 2w_1 + w_2 - 2w_3 \\
        7 = 2w_1 + 2w_2 + w_3
        \end{cases}
        \]
        
        \[
        W = \frac{1}{3} \begin{pmatrix} 29 \\ -27 \\ 17 \end{pmatrix}
        \]
        
        \section{Найти координаты $y = 3a_1 - a_2 + 7a_3$ в базисе $\{b\}$} 
        \vspace{0.5cm}
        
        \[
        C^T = \frac{1}{2} \begin{pmatrix} 
        1 & 1 & -1 \\
        1 & 3 & 5 \\
        3 & 1 & -3 
        \end{pmatrix}
        \]
        
        \[
        (C^{T})^{-1} = \frac{1}{3} \begin{pmatrix} 
        -7 & 1 & 4 \\
        9 & 0 & -3 \\
        -4 & 1 & 1 
        \end{pmatrix}
        \]
        
        \[
        X_B = (C^T)^{-1} \cdot X_A = \frac{1}{3} \begin{pmatrix} 
        -7 & 1 & 4 \\
        9 & 0 & -3 \\
        -4 & 1 & 1 
        \end{pmatrix} \cdot \begin{pmatrix} 
        3 \\ -1 \\ 7 
        \end{pmatrix}
        \]
        
        \[
        X_B = \begin{pmatrix} 2 \\ 2 \\ -2 \end{pmatrix}
        \]
        
\end{document}
